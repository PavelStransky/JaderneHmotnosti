\documentclass[a4paper,12pt,oneside]{article}

%\usepackage{palatino,newtxmath,bbold}	%% Písma
\usepackage{lmodern}
\usepackage{amsthm,thmtools}
\usepackage{bold-extra}

\usepackage{microtype}					%% Lepší mezery

\usepackage[T1]{fontenc}
\usepackage[utf8]{inputenc}	            %% Kódování textu
\usepackage[czech]{babel}               %% České nápisy
\usepackage{epsfig}
\usepackage{csquotes}

\usepackage[dvipsnames]{xcolor}
\usepackage{indentfirst}				%% Odsazení prvního odstavce

\usepackage{tablefootnote}              %% Poznámky pod čarou z tabulek

\usepackage{listings}
\usepackage{comment}

\usepackage{mhchem}                     %% Psaní značek nuklidů

\usepackage{amsmath,amsfonts,amssymb}
\usepackage[unicode]{hyperref}			%% Hypertextové odkazy
\hypersetup{
	pdftitle={Systematika jaderných hmotností},
	pdfauthor={Pavel Stránský},
	pdffitwindow=true,
	colorlinks=true,
	urlcolor=cyan,            			%barva textu pri tisku
	linkcolor=red,
	citecolor=green,
	filecolor=magenta
}

% Velikost stránky
\addtolength{\topmargin}{-1.5cm} %\addtolength{\textheight}{-10cm}
\addtolength{\textwidth}{4cm} \addtolength{\textheight}{4cm} % Šířka a výška textu
\addtolength{\voffset}{-0.5cm} % Horní okraj
\addtolength{\hoffset}{-2cm}
\setlength{\headheight}{15pt}

\renewcommand{\Im}{\mathop{\text{Im}}}
\renewcommand{\Re}{\mathop{\text{Re}}}

\pagestyle{headings} 
% \addtolength{\textwidth}{4cm} \addtolength{\textheight}{6cm} 
% \addtolength{\voffset}{-3cm} 
% \addtolength{\hoffset}{-2cm} \setlength{\parskip}{0.5ex plus 0.5ex minus 0.1ex}

\def\unit#1{\,\mathrm{#1}}
\def\c{,\!}                             % Čárka v číslech
\def\operator#1{\hat{\mathsf{#1}}}
\def\vectoroperator#1{\boldsymbol{\mathsf{\hat{#1}}}}
\def\im{\mathrm{i}}
\DeclareMathOperator{\e}{e}

\newtheoremstyle{red}
{5pt}{5pt}{\itshape\color{red}}{}{\bfseries\color{red}}{:}{.5em}{}

\theoremstyle{red}
\declaretheorem[name=Úkol,numberwithin=section]{task}

\begin{document}
\title{Systematika jaderných hmotností}
\date{\today}
\author{Pavel Stránský}

\maketitle

\section{Jaderné jednotky}
V jaderné a částicové fyzice se využívá Einsteinem odvozené ekvivalence mezi hmotností $M$ a energií $E$
\begin{equation}
    E=Mc^{2}
\end{equation}
kde $c\approx3\c0\cdot10^8\unit{ms^{-1}}$ je rychlost světla.
Za jednotku energie se bere \emph{elektronvolt}
\begin{equation}
    [E]=\mathrm{eV}=1\c602\cdot10^{-19}\unit{J},
\end{equation}
což, jak název napovídá, je kinetická energie elektronu urychlená ve vakuu napětím $1\,\mathrm{V}$.
Navíc se pracuje v tzv.~\emph{přirozených jednotkách}, ve kterých je rychlost světla $c=1$, a tudíž hmotnost i energie mají stejnou jednotku.

\section{Typické objekty mikrosvěta}
\begin{minipage}{0.95\linewidth}
\begin{tabular}{|l|l|l|c|l|}
    \hline
    & & {\color{red}\bf hmotnost} & náboj & poločas rozpadu \\
    \hline
    \emph{elektron} & $e^{-}$ & $M_{e}=0\c511\unit{MeV}$ & $-$ & stabilní \\
    \emph{proton} & $p^{+}$ & {\color{red}$M_{p}=938\unit{MeV}$} & $+$ & stabilní\footnote{
        Některé teorie předpovídají, že proton se rozpadá s enormním poločasem rozpadu $T_{p}\approx10^{32}\unit{let}$.
        } \\
    \emph{neutron} & $n^{0}$ & {\color{red}$M_{n}=940\unit{MeV}$}\footnote{
        Neutron je těžší než proton. Volný neutron se rozpadá procesem
        \begin{equation}
            \label{eq:beta}
            n^{0}\rightarrow p^{+}+e^{-}+\overline{\nu}_{e},
        \end{equation}
        kde $\overline{\nu}_{e}$ je elektronové antineutrino.
        } & $0$ & $T_{n}=611\unit{s}$ \\
    \hline
\end{tabular}
\end{minipage}

\begin{task}
    Převeďte hmotnosti z tabulky na kg.
\end{task}

\section{Atomové jádro}
    Zde budeme uvažovat pouze atomové jádro, nebudeme se tedy zabývat elektrony z elektronového obalu atomu.

    Jádro se skládá z protonů a neutronů. Ty se souhrnně označují jako \emph{nukleony}.
    Počet protonů $Z$ se nazývá \emph{protonové číslo},
    počet neutronu $N$ se nazývá \emph{neutronové číslo}
    a jejich součet $A=N+Z$ je \emph{hmotnostní číslo} nebo nukleonové číslo.

    Jádro se obecně označuje $\ce{^{\it A}_{\it P}X}$, kde X je symbol daného prvku, například $\ce{^{235}_{92}U}$.
    Protonové číslo se často vynechává, protože je jednoznačně dáno symbolem prvku.

    Hmotnost atomového jádra je
    \begin{equation}
        M=ZM_{p}+NM_{n}-B,
    \end{equation}
    kde $B$ je \emph{vazebná energie}.
    Aby jádro mohlo existovat, musí být $B>0$.
    Čím je vazebná energie vyšší, tím je jádro silněji vázané a stabilnější.
    Další často používaná veličina je vazebná energie na nukleon,
    \begin{equation}
        b=\frac{B}{A}.
    \end{equation}

\section{Jaderná interakce}
    Nukleony se drží v jádře díky silné jaderné interakci. 
    Ta je krátkodosahová, přitažlivá a působí přibližně stejnou silou mezi protony i mezi neutrony.

\section{Kapkový model jádra}
    Jeden z nejjednodušších modelů atomového jádra aproximuje jadernou hmotu jako \emph{nestlačitelnou} \emph{homogenně nabitou} kapalinu.
    Poloměr jádra je tak
    \begin{equation}
        R=R_{0}\sqrt[3]{A},
    \end{equation}
    kde $R_{0}=1\c2\unit{fm}$.\footnote{Jednotka femtometr $\mathrm{fm}=10^{-15}\unit{m}$ se obvykle v jaderné fyzice nazývá fermi.}
    Vazebná energie je pak dána výrazem
    \begin{equation}
        \label{eq:BetheWeizsackerVS}
        B=a_{V}'V-a_{S}'S-\frac{3}{5}\frac{1}{4\pi\epsilon_{0}}\frac{Z^{2}}{R}-a_{A}\frac{\left(N-Z\right)^{2}}{A}+\delta(N,Z).
    \end{equation}

    \begin{itemize}
        \item\emph{Objemový člen.}
            První člen vyjadřuje interakci jednotlivých nukleonů se svými nejbližšími sousedy a je úměrný objemu jádra
            \begin{equation}
                V=\frac{4}{3}\pi R^{3}.
            \end{equation}
    
        \item\emph{Povrchový člen.}
            Druhý člen snižuje vazebnou energii tím, že nukleony na povrchu jádra interagují s méně nukleony než ty uvnitř.
            Odpovídá povrchovému napětí u klasických kapalin a energie s ním spojená je úměrná povrchu jádra
            \begin{equation}
                S=4\pi R^{2}.
            \end{equation}

        \item\emph{Coulombický člen.}
            Třetí člen vyjadřuje elektrostatické odpuzování kladně nabitých protonů.
            Odpovídá potenciální energii homogenně nabité koule.

        \item\emph{Asymetrie (Pauliho člen).}
            Pro jádro je výhodnější, když má stejný počet protonů a neutronů.
            Je to důsledek Pauliho vylučovacího principu: 
            Protony a neutrony jsou \emph{fermiony}, které se nemohou vyskytovat ve stejném stavu, a tudíž se nemohou nacházet na stejném místě.
            Čím více protonů nebo neutronů, tím musejí být dál od středu jádra, a jsou tedy slaběji vázané.

        \item\emph{Párovací člen.}
            Pro fermiony je mnohem energeticky výhodnější, pokud se vyskytují v párech.
            Vazebná energie je tedy vyšší, pokud má jádro sudý počet protonů i neutronů (sudo-sudá jádra) než pokud je $N$ nebo $Z$ liché.
            Licho-lichá jádra jsou nejvíce nestabilní a mají tendenci přeměnit jeden nukleon v opačný typ, čímž se jejich stabilita výrazně zvýší.
            V přírodě se vyskytuje jen pět stabilních licho-lichých jader, a to čtyři lehká jádra (zde by přeměna $n^{0}\leftrightarrow p^{+}$ vedla k výrazné asymetrii počtu protonů a neutronů)
                $\ce{^{2}_{1}H}$, 
                $\ce{^{6}_{3}Li}$, 
                $\ce{^{10}_{5}B}$ a
                $\ce{^{14}_{7}N}$,
            a jedno těžké jádro, izomer $\ce{^{180}_{73}Ta}$, které však může mít tak dlouhý poločas rozpadu, že jeho rozpad zatím nebyl pozorován.
        \end{itemize}

    Zatímco prvním třem členům lze porozumět pomocí klasické fyziky, Pauliho člen a párovací člen se nedají vysvětlit jinak než pomocí mikroskopické teorie (kvantové teorie).

    Vzorec~\eqref{eq:BetheWeizsackerVS} se obvykle užívá ve tvaru
    \begin{equation}
        \label{eq:BetheWeizsacker}
        B=a_{V}A-a_{S}A^{\frac{2}{3}}-a_{C}\frac{Z^{2}}{R}-a_{A}\frac{\left(N-Z\right)^{2}}{A}+\delta(N,Z).
    \end{equation}
    a nazývá se \emph{Betheho-Weizsäckerova formule}.
    Parametry $a_{V},a_{S},a_{C},a_{A}$ a $\delta$ jsou všechny kladné a určují se z naměřených dat známých jader.

    Jádro má ve skutečnosti slupkovou strukturu podobně jako atomový obal.

\section{Tabulka nuklidů}
    Atomová jádra s daným $N$ a $Z$ se nazývají \emph{nuklidy}.
    Uspořádávají se do ~\emph{tabulky nuklidů}\footnote{Nuclear chart.}, kde na vodorovné ose je neutronové číslo $N$ a na svislé ose protonové číslo $Z$.
    Do tabulky se běžně uvádí způsoby přeměny (rozpadové kanály), poločas přeměny (střední doba života) a další charakteristiky atomového jádra.
    Jádra stejného prvku lišící se počtem neutronů se nazývají \emph{izotopy}.

    \subsection{Radioaktivní přeměna prvků}
        Atomová jádra mají tendenci přeměňovat se tak, aby byla co nejstabilnější. Tím se zvyšuje jejich vazebná energie a při přeměně se přebytečná energie vyzařuje, což se využívá například v energetice v atomových elektrárnách.
        Z Betheho-Weizsäckerovy formule je vidět, že kdyby nebylo Coulombického odpuzování, pak by vždy nejstabilnější jádra měla díky asymetrickému členu $N=Z$.        
        Coulombické odpuzování způsobuje, že stabilnější jádra budou mít více neutronů než protonů, přičemž tento efekt je výraznější u těžších jader.
        Z formule~\eqref{eq:BetheWeizsacker} lze odvodit, že nejstabilnější jádra mají 
        \begin{equation}
            \frac{N}{Z}\approx 1+\frac{a_{C}}{2a_{A}}A^{\frac{2}{3}},
        \end{equation}
        tj. \emph{údolí stability} se v tabulce nuklidů otáčí pod diagonálu.

        Tři nejdůležitější procesy přeměny jsou
        \begin{enumerate}
            \item\emph{$\alpha$ rozpad.}
                Z jádra vylétne $\alpha$ částice, což je velmi silně vázané jádro $\ce{^{4}_{2}He}$,
                \begin{equation}
                    \ce{^{\it A}_{\it Z}X}\rightarrow\ce{^{{\it A}-4}_{{\it Z}-2}Y}+\ce{^{4}_{2}He}.
                \end{equation}
                Tento proces je neběžnější pro těžká jádra v oblasti údolí stability a nad ním.
                V tabulce nuklidů získáme procesem jádro posunuté diagonálně o dvě políčka vlevo dolů.

            \item\emph{$\beta$ rozpad.}
                V jádře se přemění neutron na proton procesem~\eqref{eq:beta}.
                Hmotnostní číslo tedy zůstane zachováno, z jádra vylétne vzniklý elektron (a antineutrino),
                \begin{equation}
                    \ce{^{\it A}_{\it Z}X}\rightarrow\ce{^{\it A}_{{\it Z}+1}Y}+e^{-}+\bar{\nu}_{e}.
                \end{equation}
                Nové jádro se nachází o jedno políčko diagonálně vlevo nahoru.
                $\beta$ rozpad je nejčastější u jader s přebytkem neutronů, což jsou jádra nacházející se pod údolím stability.
            
            \item\emph{$\beta^{+}$ rozpad.}
                V jádře se přemění proton na neutron.
                jedná se o proces opačný k $\beta$ rozpadu.
                Z jádra vylétne pozitron (a neutrino),
                \begin{equation}
                    \ce{^{\it A}_{\it Z}X}\rightarrow\ce{^{\it A}_{{\it Z}-1}Y}+e^{+}+\nu_{e}.                    
                \end{equation}
                Nové jádro se nachází o jedno políčko diagonálně vpravo dolů.
                Tato přeměna se vyskytuje u jader s přebytkem protonů (jádra nad údolím stability).
        \end{enumerate}

        Kromě těchto procesů mohou ještě jádra přímo vystřelit proton, neutron, nebo se samovolně rozštěpit na dvě lehčí jádra.
        K těmto procesům však dochází jen u extrémních případů nuklidů vyrobených uměle.

    \subsection{Rozpadové řady}
        Produkty přeměn nestabilních nuklidů mohou opět podléhat radioaktivním přeměnám.
        Ze směrů rozpadů v tabulce nuklidů je jasné, že celá tabulka se rozpadá na čtyři nezávislé skupiny nuklidů.
        Rozpadové řady začínající u v přírodě se vyskytujících nuklidů ze skupiny aktinoidů a končící stabilním prvkem, nejčastěji olovem, jsou:
        \begin{itemize}
            \item $\ce{^{238}U}\rightarrow\ce{^{206}Pb}$
            \item $\ce{^{235}U}\rightarrow\ce{^{207}Pb}$
            \item $\ce{^{232}Th}\rightarrow\ce{^{208}Pb}$
            \item $\ce{^{237}Np}\rightarrow\ce{^{205}Tl}$
        \end{itemize}
        \begin{task}
            Pomocí tabulky nuklidů určete všechny meziprodukty těchto rozpadových řad.
        \end{task}

\section{Kompilát jaderných hmotností}
    Jaderné hmotnosti měří laboratoře po celém světě a své výsledky průběžně publikují v odborných časopisech.
    Přibližně jednou za čtyři roky je vydán souhrnný článek se všemi doposud změřenými hmotnostmi včetně experimentálních odchylek.
    Poslední vydanou sbírkou je \href{https://www-nds.iaea.org/amdc/}{Atomic Mass Evaluation -- AME 2020}.
    Obsahuje hmotnosti 2550 měřených nuklidů.
    Sbírka je veřejně dostupná a skládá se z odborných článků shrnujících použité experimentální metody a diskutujících přesnost měření a rozdíly oproti předchozím sbírkám, a z textového souboru se syrovými daty.
    Analýza textového souboru bude hlavní náplní tohoto projektu.

    \begin{task}
        Stáhněte si soubor s jadernými hmotnostmi a vytvořte jednoduchý program, který ze souboru načte k jednotlivým nuklidům jejich vazebné energie.
    \end{task}

    Přesné hodnoty hmotností pro výpočty:
    \begin{align}
        M_{p}&=938\c2720882\unit{MeV},\\
        M_{n}&=939\c5654205\unit{MeV},\\
        u=\frac{1}{12}M(\ce{^{12}C})&=931\c4941024\unit{MeV}.
    \end{align}

    \begin{task}
        Vykreslete graf vazebné energie pro všechny nuklidy.
    \end{task}

    \begin{task}
        Nalezněte nuklidy s nejvyšší vazebnou energií na nukleon.
    \end{task}

    \begin{task}
        Určete, která jádra se mohou rozpadat $\alpha$ rozpadem a zakreslete je do grafu.
    \end{task}

    \begin{task}
        Určete, která jádra se mohou rozpadat $\beta$ rozpadem a $\beta^{+}$ rozpadem a zakreslete je do grafu.
    \end{task}

    \begin{task}
        Nalezněte parametry rovnice~\eqref{eq:BetheWeizsacker}.
    \end{task}

    \begin{task}
        Určete dvouneutronovou separační energii
        \begin{equation}
            S_{2n}=M(Z,N-2)+2M_{n}-M(Z,N)
        \end{equation}
        a zakreslete ji do grafu.
    \end{task}

\end{document}